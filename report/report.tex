\documentclass[letterpaper]{article}
\usepackage{natbib,alifeconf}
\newcommand\tab[1][1cm]{\hspace*{#1}}

\usepackage{xcolor}
\usepackage{amsmath}

\makeatletter
\def\mcolor#1#{\@mcolor{#1}}
\def\@mcolor#1#2#3{%
  \protect\leavevmode
  \begingroup
    \color#1{#2}#3%
  \endgroup
}
\makeatother

\newcommand{\annotate}[3]{
\mcolor{#1}{\overbrace{#3}^\text{#2}}
}

\newcommand{\sitem}[1]
{
    \begin{itemize}
        \item #1
    \end{itemize}
}

\definecolor{orange}{HTML}{f7c767}
\definecolor{blue}{HTML}{67E6F7}
\definecolor{green}{HTML}{bdf767}
\definecolor{purple}{HTML}{f467f7}
\definecolor{red}{HTML}{fc4475}

\usepackage{tikz}
\usetikzlibrary{arrows.meta}

\title{Report for Complex Adaptive Systems}
\author{Lucas Saldyt}

\begin{document}
\maketitle

\begin{abstract}
    Ants of the genus Temnothorax display the captivating ability to choose optimally between nest sites in an entirely decentralized, error correcting, and self-organizing process.
    First, this paper reviews previous models of Temnothorax decision making, and simplifies and then analyzes them.
    Specifically, it shows that previous agent based models can be expressed with far fewer states and transitions. 
    It also extends previous models of population dynamics.
    Secondly, this paper aims to evaluate the applications of this behavior to computer science.
\end{abstract}

\section{Introduction}

%% Temnothorax nest choice is worth studying because decentralized and self-organizing algorithms are  

\subsection{Summary (Mallon 2001)}

Nest choice is deeper than may initially seem.
First of all, Temnothorax Albipennis colonies generally have relatively small sizes , which in theory limits decision making ability, but in practice does not [cite mallon 2002]. 
Ants are able to use graded assessment to make decisions efficiently even though their colony size is small.

Additionally, a small number of comparisons are actually made between colonies. 
Mallon 2001 publishes three experiments with $86\%, 46\%$ and $32\%$ direct comparison, which indicates the presence of decentralized behavior.
Ants seem to use peer rate estimation to decide on the quality of a particular colony: instead of counting the number of ants at a colony, they estimate the total number from the frequency of ants (if many ants are seen over a brief period of time, then an ant knows that the current location has a large number of ants).

%% \begin{itemize}
%%     \item How does small colony size affect decision making?
%%         \sitem{\mcolor{red}{Graded assessment} affects the colony feedback loop}
%%     \item What proportion of comparison is by individuals as opposed to at the colony level? 
%%     \sitem{\mcolor{red}{$86\%, 46\%, 32\%$} for different colonies.}
%%     \sitem{Direct comparison is \mcolor{red}{not crucial} for choosing the optimal site and therefore the decision is \mcolor{red}{decentralized}}
%%     \item What physical observations act as cues for behavior? 
%%         \sitem{\mcolor{red}{Nestmate interaction rate} for estimation of colony/environment state}
%% \end{itemize}

When deciding on nests, ants actually use a variety of different recruiting mechanisms: Direct tandem runs (leading of other ants), transportations (carrying of other ants), and reverse tandem runs (leading of other ants, but in the opposite direction). 
Early work set out to describe the purpose and importance of each of these mechanisms.

\subsection{Summary (Pratt 2002, 2005)}

Each recruitment mechanism has different advantages. Tandem running allows learning of the route to a nest as well as the deposition of pheromones along the route. 
Later in the decision process, ants switch to ``transport`` recruitment, where they literally carry other ants. 
This mechanism triggers when ants know that a destination nest has a large number of ants (is above the quorum, or threshold, which is estimated by encounter rate).

Faster recruitment to better sites allows decentralized optimal choice without direct comparison.
In other words, when an ant encounters a good site, it recruits to that site very quickly, which causes positive feedback when subsequent ants encounter and recruit to the same site.
This can be seen in the population equations from Pratt 2002, where incoming ants depend on the number of recruiting ants.

%%        \begin{itemize}
%%            \item What is the purpose of multiple recruitment mechanisms? (Pratt 2002, 2005)
%%                \sitem{Tandem runs allow route learning and pheromone depositing}
%%                \sitem{\mcolor{red}{Graded commitment} allows error-correction and optimal decision making.}
%%                \sitem{Recruitment \mcolor{red}{accelerates} when transport begins ($3x$ speed)}
%%            \item How do individuals choose between recruitment mechanisms? (Pratt 2005)
%%                \sitem{Transports preffered once \mcolor{red}{quorum} is met}
%%            \item How does decentralization happen without direct comparison? (Pratt 2005)
%%                \sitem{Recruit to better sites faster $\rightarrow$ \mcolor{red}{positive feedback}}
%%        \end{itemize}

Reverse tandem runs had no single explaining mechanism, but it was hypothesized that they either stimulated transport by idle workers, or fixed nest-splitting that would be more common in nature than in the lab.
The quorum requirement seems to assist the ants in making optimal choices by acting as a general error correction mechanism --- it delays decision making in case ants have chosen a sub-optimal nest, and this decreases the likelihood of colony splitting.

\subsection{Summary (Granovskiy 2012)}

Granovskiy 2012 simplifies many of the ideas in the previous 2005 agent-based model.
It still has four macro-states: Exploring, Assessing, Canvassing (Leading), and Committed (Carrying).
However, it is simplified each of them so that they contain only the substates: search and at-nest, as well as their respective specialized actions (tandem runs for the canvassing population, and transport and reverse tandem runs for the committed populations). 
Also, assessing ants can begin recruiting once they accept a nest.

Additionally, there is the possibility that any searching ant can be picked up and carried to a nest nest, and any ant can be led by tandem run.
Otherwise, this model does not have any extra features from the 2005 model, but still seems to perform similarly.


%%       \begin{itemize}
%%               \sitem{Selection of an individual ant's ``home`` nest}
%%           \item How is quorum detected?
%%               \sitem{Ants likely use \mcolor{red}{rate estimation} rather than literal counting}
%%           \item \mcolor{red}{\em{Important parts of the decision process occur at the individual level}}
%%       \end{itemize}
%% \section{Previous models}
%% 
%%   \subsection{2002 Model Overview}
%%   S.Pratt 2002 contains an ODE model, which models a colony of $N$ ants, where proportion $p$ are active. \\
%%       $M$ sites (in practice $~2-12$ sites). \\
%%       Four population classes: 
%%       \begin{itemize}
%%           \item $S$: Searching (At destroyed nest or between nests) \mcolor{black}{(Initially $Np$)}
%%           \item $A_i$: Assessing a nest $i$ \mcolor{black}{(Initially 0)}
%%           \item $R_i$: Recruiting to a nest $i$ \mcolor{black}{(Initially 0)}
%%           \item $P_i$: The passive population at nest i \mcolor{black}{($P_0 = N(1-p), P_i = 0$)}
%%       \end{itemize}
%% 
%%   \subsection{Overview of Pratt 2002 Equations (Colors showing flow of ant population)}
%%       \Large
%%       \begin{equation}
%%       \begin{aligned}
%%           & \frac{ds}{dt} = - \sum_{j=1}^m \mcolor{black}{u_js} - \sum_{j=1}^m \mcolor{black}{\lambda_ji(r_j, s)} \\
%%           & \frac{da_i}{dt} = \mcolor{black}{u_is} + \mcolor{black}{\lambda_ji(r_j, s)} + \sum_{j \neq i}^m \mcolor{black}{(\rho_{ji}a_j - \rho_{ij}a_i)} - \mcolor{black}{k_ia_i} \\
%%           & \frac{dr_i}{dt} = \mcolor{black}{k_ia_i} + \sum_{j \neq i}^m \mcolor{black}{(\rho_{ji}r_j - \rho_{ij}r_i)} \\
%%           & \frac{dp_i}{dt} = \annotate{black}{causes overflow if unchecked}{\phi_ij(r_i, p_0)} - \annotate{black}{fix: subtract moved ants}{\phi_{dest}j(r_{dest},p_0)}
%%       \end{aligned}
%%       \end{equation}
%%   \subsection{Search population}
%%       \Large
%%       \begin{multline}
%%       \frac{dS}{dt} = - \sum_{j=1}^M \annotate{black}{Rate of finding nest $j$}{u_j}*S \\  
%%       - \sum_{j=1}^M \annotate{black}{Ants led by tandem run}{\lambda_jI(R_j, S)} 
%%       \end{multline}
%%   \subsection{Assessment population}
%%       \Large
%%       \begin{multline}
%%       \frac{dA_i}{dt} = \annotate{black}{Ants that find nest $i$ and begin assessing it}{u_iS} \\ 
%%       + \annotate{black}{Ants led by tandem run}{\lambda_jI(R_j, S)} \\
%%       + \sum_{j \neq i}^M \annotate{black}{Ants encountering alternative sites}{(\rho_{ji}A_j - \rho_{ij}A_i)} \\ 
%%       - \annotate{black}{Ants that become recruiters}{k_iA_i} \\
%%       \end{multline}
%%   \subsection{Recruitment population}
%%       \Large
%%       \begin{multline}
%%           \frac{dR_i}{dt} = \annotate{black}{Ants that become recruiters}{k_iA_i} \\
%%           + \sum_{j \neq i}^M \annotate{black}{Ants encountering alternative sites}{(\rho_{ji}R_j - \rho_{ij}R_i)}
%%       \end{multline}
%%   \subsection{Passive population}
%%       \Large
%%       \begin{equation}
%%           \frac{dP_i}{dt} = \underbrace{\annotate{black}{Per capita transport rate}{\phi_i} * \annotate{black}{Quorum requirement}{J(R_i, P_0)}}_\text{Carried passive ants}
%%       \end{equation}
%%   \subsection{Auxilary Functions}
%%       \Large
%%       Tandem/Carrying switching rule:
%%       \begin{equation}
%%           I(R_i, S) = 
%%           \begin{cases}
%%               & R_i,  \text{ if } \annotate{black}{Recruiters below threshold}{R_i < T} \\
%%               &       \text{     and } \annotate{black}{While there are recruitable ants}{S > 0}\\
%%               & 0, \text{ otherwise}
%%           \end{cases}
%%       \end{equation}
%% 
%%   \subsection{Auxilary Functions}
%%       \Large
%%       Quorum rule:
%%       \begin{equation}
%%           J(R_i, P_0) = 
%%           \begin{cases}
%%               & 0,  \text{ if } \annotate{black}{Recruiters below threshold}{R_i < T} \\  
%%               &     \text{     or } \annotate{black}{Edge case: migration has finished}{P_0 = 0}\\
%%               & R_i, \text{ otherwise}
%%           \end{cases}
%%       \end{equation}
%% 
%%   \subsection{Assumptions}
%%       \begin{itemize}
%%           \item Ants always move exclusively out of the searching state
%%           \item Finding/Recruitment/Switching/Conversion rates are all constant per nest
%%           \item Requires detailed parameter estimation
%%       \end{itemize}
%% 
%%   \subsection{Diagram of ODE model (Pratt 2002)}
%% 
%%   \subsection{The agent based model (Pratt 2005)}
%% 
%%   \subsection{Important differences between the models (In order of subjective importance)}
%%       \begin{itemize}
%%           \item Ants remember their ``home`` nest.
%%           \item Commitment level is separated from individual actions 
%%               \sitem{(i.e. an ant can follow even in the committed state)}
%%               \sitem{Also allows for further parameterization}
%%           \item Intentionally makes the model more brittle so that it can be further tested empirically. 
%%               \sitem{A general model is potentially harder to disprove}
%%           \item Adds a simple ``at-nest`` action 
%%           \item Minute differences (potential for ants to get lost)
%%       \end{itemize}
%% 
  \section{Proposed Ordinary Differential Equation Model}

  The proposed model begins with an improved set of ordinary differential equations, based on Pratt 2002.
  It contains equations for five separate populations: 
  \begin{itemize}
      \item $S$, the searching population (not at any nest)
      \item $A_i$, the assessing population at nest $i$
      \item $L_i$, the leading (forward-tandem-running) population at nest $i$
      \item $C_i$, the carrying (transport) population at nest $i$
      \item $P_i$, the passive population at nest $i$. 
  \end{itemize}
  The model focuses on the following:
  \begin{itemize}
      \item Splitting the $R_i$ population from S. Pratt 2002 into the $L_i$ and $C_i$ populations.
      \item Replacing the two switching equations $I()$ and $J()$ with dynamics switching between $L_i$ and $C_i$ based on a single switching equation $Q()$.
      \item Fixing unchecked growth in the original $P_i$ equations.
      \item Allowing transport of various passive populations, which will allow a split passive population to be fixed.
      \item Replacing switching in $A_i$ and $R_i$ with transitions to searching population. This reflects updates in S.Pratt 2005 and Granovskiy 2012 agent-based models.
      \item Adding transportation of the active searching population (but not the assessing, leading, or carrying populations).
  \end{itemize}

  Given $N$ ants, where proportion $p$ are active, the initial states are the following:
  \begin{itemize}
      \item $S = pN$
      \item $P_0 = (1-p)N$ 
      \item $A_i, L_i, C_i, P_i = 0$ 
  \end{itemize}

  The original model used the following parameters:\\

\begin{tabular}{ l | r }
    \hline
  $\mu_i$     & Likelihood of finding nest $i$\\
  $\lambda_i$ & Proportion led by leaders to $i$\\
  $\rho_{ij}$ & Switching rates between nests $i$ and $j$\\
  $k_i$       & Acceptance probability for nest $i$\\
  $\phi_i$    & Rate for carrying passive ants to nest $i$\\
    \hline
\end{tabular} \\

The updated model builds on this list, but renames old parameters to make them more intuitive: \\

\begin{tabular}{ l | c | r }
    \hline
  Name          & 2002        & Description (units = rate)\\ \hline
  $\phi_i$      & $\mu_i$     & Finding nest $i$\\
  $\lambda_i$   & $\lambda_i$ & Led by leaders to $i$\\
  T             & T           & Threshold (positive integer)\\
  $\alpha_i$    & $k_i$       & Assessors who accept nest $i$\\ \hline
  $\tau_{Pi}$   & $\phi_i$    & Passive ants are transported to $i$\\
  $\tau_{Si}$   & \em{New}    & Searching ants are transported to $i$\\ \hline
  $\sigma_{Ai}$ & \em{New}    & Assessing ants enter search from $i$\\
  $\sigma_{Li}$ & \em{New}    & Leading ants enter search from $i$\\
  $\sigma_{Ci}$ & \em{New}    & Carrying ants enter search from $i$\\
  \hline
\end{tabular} \\

%% For clarity, the following table relates parameters in this model to the parameters in [TODO: Cite Spratt 2005 and grav 2012]\\
%% 
%% \begin{tabular}{ l | r }
%%     \hline
%%     Name & 2005 \\ \hline
%%   $\phi_i$      &  \\
%%   $\lambda_i$   &  \\
%%   T             &  \\
%%   $\alpha_i$    &  \\ \hline
%%   $\tau_{Pi}$   &  \\
%%   $\tau_{Si}$   &  \\ \hline
%%   $\sigma_{Ai}$ &  \\
%%   $\sigma_{Li}$ &  \\
%%   $\sigma_{Ci}$ &  \\
%%   \hline
%% \end{tabular} \\
\subsubsection{Parameter Descriptions} 

$\phi_i$, previously $\mu_i$, describes the rate at which search ants find nest $i$. 
Generally, this will be used to describe nests that are at different distances from the original destroyed nest. \\
$\lambda_i$ describes the rate at which ants are led by leaders to nest $i$. 
Differences in $\lambda_i$ would describe nests which were led to faster. \\
$T$ is simply the quorum threshold, and is a free variable meant to be experimented with.
Pratt 2002 found that a value between $8$ and $30$ was best. \\
$\alpha_i$, previously $k_i$, describes the rate at which assessors accept a nest and begin recruiting.
It is this parameter which allows a rapid positive feedback loop to occur, and this is largely responsible for optimal nest choice. \\
$\tau_{Pi}$, previously $\phi_i$ is the rate at which passive ants are transported to nest $i$.
However, this model introduces a few new parameters, based on the agent based models in [TODO: Gravinvosky?] and Pratt 2005. 
For instance, $\tau_{Si}$ describes the transport of searching ants. 
TODO: Should there also be transport of other active ants? \\
$\sigma$ describes the rate at which ants in an active state ($A_i$, $L_i$, or $C_i$) enter searching again.
For instance, $\sigma_{Ai}$ would denote the rate at which assessing ants enter search from nest $i$.

\subsubsection{Proposed Equations and Descriptions}

The following equation describes the searching population, which starts as $Np$.
\begin{multline}
    \frac{dS}{dt} = \sum \big[ -\phi_iS - \lambda_iL_iS - \tau_{Si}C_iS \\ + \sigma_{Ai}A_i + \sigma_{Li}L_i + \sigma_{Ci}C_i \big]
\end{multline}

The first term, $\phi_iS$, describes ants that encounter new sites and enter the assessment population.
$\lambda_iL_iS$ describes ants being led to new sites and becoming assessors. $L_i$ is included here because the presence of more leading ants will increase the rate at which ants are led to new sites (i.e. ten leading ants lead ants faster than a single ant). Therefore $\lambda_i$ is proportional to $L_i$.
$\tau_{Si}C_iS$ describes ants that are transported to new sites. As with the previous term, $\tau_{Si}$ is a rate per individual in $C_i$.
Lastly, the $\sigma$ terms describe ants that exit other active states and begin searching, each with an independent rate.

The following equation describes the assessment populations, which start at $0$:
\begin{multline}
    \frac{dA_i}{dt} = \phi_iS + \lambda_iL_iS + \tau_{Si}C_iS - \sigma_{Ai}Ai - \alpha_iA_i
\end{multline}

The first three terms match the first three terms of the search-population equation.
$\phi_iS$ describes incoming ants that have found the nest themselves, $\lambda_iL_iS$ describes ants that were carried to the nest $i$, and $\tau_{Si}C_iS$ describes ants that were carried to the nest $i$.

$\sigma_{Ai}A_i$ describes ants that begin searching after assessing a nest, and lastly $alpha_iA_i$ describes ants that accept a nest and begin recruiting.

The following equation describes the leading populations, which starts at $0$.

\begin{multline}
    \frac{dL_i}{dt} = (1-Q(i))\alpha_iA_i - Q(i)L_i + (1 - Q(i))C_i - \sigma_{Li}L_i
\end{multline}

First, the function $Q()$, defined below, returns $1$ if the nest $i$ is above the quorum threshold and $0$ otherwise. Therefore $1 - Q(i)$ is $0$ when the nest is above the quorum threshold and $1$ otherwise.
So, when the nest $i$ is above the quorum threshold, only the terms $-Q(i)L_i$ and $\sigma_{Li}L_i$ are active. $Q(i)L_i$ represents a movement of ants from leading to carrying. $\sigma_{Li}L_i$ represents leading ants deciding to enter the search state.
When the nest is below the quorum threshold, the first and third term are both active. $1-Q(i)\alpha_iA_i$ describes assessing ants entering the leading populations after accepting a nest, and $1-Q(i)Ci$ describes (potentially) carrying ants reverting to the leading state.

The following equation describes the carrying populations, which starts at $0$:

\begin{multline}
    \frac{dC_i}{dt} = Q(i)\alpha_iA_i - (1 - Q(i))C_i + Q(i)L_i - \sigma{Ci}C_i
\end{multline}

Similarly to the leading population equations, $Q()$ acts as a switch. When the nest $i$ is above the quorum threshold, assessing ants will enter the carrying population directly through the $Q(i)alpha_iA_i$ term and leading ants are converted through the $Q(i)L_i$ term.
The term $(1-Q(i))C_i$ describes ants that (potentially) revert to the leading state. 
Lastly, the term $\sigma{Ci}C_i$ describes ants that enter the searching state from carrying.

The following equation describes the passive population dynamics. Initially, $P_0 = (1-p)N$ and otherwise $P_i = 0$:
\begin{equation}
\begin{aligned}
  \frac{dP_i}{dt} = \sum_{j \neq i} [\tau_{Pi}P_jC_i - \tau_{Pj}P_iC_j]
\end{aligned}
\end{equation}

Essentially, the first term $\tau{Pi}P_jC_i$ desribes ants being moved to site $i$ from site $j$, while the second term describes ants being moved from site $i$ to site $j$.

The $Q()$ switching function is defined as follows, and should be self explanatory. Potentially, (TODO) $P_i$ population should not be counted (but in simulation, this is not a contributing factor):

\begin{equation}
\begin{aligned}
  & Q(i) = 0, \text{if    } \sum [A_i + L_i + C_i + P_i] \leq T \\
  & Q(i) = 1, \text{otherwise} \\
\end{aligned}
\end{equation}

These equations do not account for reverse tandem runs, but could be modified easily to do so (especially once tandem runs are better understood).

  \subsection{Bibliography}
      \bibliography{sources/insects,sources/brains,sources/ant_system,sources/networks,sources/misc}

\footnotesize
\bibliographystyle{apalike}
\bibliography{sample}


\end{document}
