\documentclass{beamer}
\usetheme{owl}           % Use metropolis theme
\title{Quantifying and Mimicking Networked Behavior in Social Insects}
\date{\today}
\author{Lucas Saldyt}
\institute{Arizona State University}

\begin{document}
  \maketitle
  \begin{frame}{Problem}
      \begin{enumerate}
      Ant colonies are capable of rudimentary problem solving.
      Problem solving in ant colonies is fundamentally different than classical conceptions of problem solving.
      Ant colonies act as a distributed but dense network of independent agents.
      At times, the colony can act independently.
      At other times, the colony cooperates as a whole.
      This emergence of organization is extremely interesting:
      How does the signal propogate?
      When does it propogate?
      Is this method of co-operation more useful? Why/Why not?
      \end{enumerate}
       
  \end{frame}
  \begin{frame}{Motivation}
      Properties of ant colonies are likely present in other systems
      Success has been had in designing algorithms based on the analogy of an ant colony
      Networks in general are everpresent in todays society
      (Networks are one of the most general data structures: understanding them is crucial)
  \end{frame}

  \begin{frame}{History}
      Ant systems have been successfully studied in the past, and used to write computer algorithms.
      %% In general, everything is biomimicry.
      A certain class of cognitive architectures acts much like an ant colony
      codelets = ants
      castes = codelet types
      workspace, other structures = colony components & environment
      Specifically

      Let us focus on the method by which ants construct their colonies, distributedly.
      Ants are capable of doing so without using a blueprint by [ ](research).
      This is of particular interest because of the idea of generative grammars.
      Given [] pieces, build [] that does [].
  \end{frame}
  \begin{frame}{Levels}
      Ants themselves have brains..
      Which are themselves nonlinear
      And the neurons in their brains.. are also nonlinear..
      So: composition and emergence? What's interesting here?
      Is it ever appropriate to assume that a level can be approximated linearly? 
      %% (Are there hofstadterian symbolic layers?)
      How close can an ant-model approximate the behavior of an ant colony?
      Is it completely necessary to model everything (say, down to the level of quarks!?)
      Micro vs Macro predictions and inevitability: like an ideal gas.
      Even if we can't predict the locations and velocities of individual ants, we can still predict other factors, especially if they are statistically derived summary factors (like a metric quantifying alertness)
  \end{frame}
\end{document}
