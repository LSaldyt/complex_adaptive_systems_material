%% Template for a preprint Letter or Article for submission
%% to the journal Nature.
%% Written by Peter Czoschke, 26 February 2004
%% Edited by Lucas Saldyt, September 11th, 2018

\documentclass{nature}

%% make sure you have the nature.cls and naturemag.bst files where
%% LaTeX can find them

\bibliographystyle{naturemag}

\title{Biomimicry of Distributed Cognition in Social Insects}

\author{Lucas Saldyt}

\begin{document}

\maketitle

\begin{affiliations}
 \item Arizona State University
\end{affiliations}

\begin{abstract}
    This paper reviews the degree of distribution\cite{saramaki2007generalizations} within ant colonies, and the application of agent-distributed behavior to artificial intelligence \cite{brabazon2016distributed, dorigo2011ant}.
    Ants at some times cooperate to accomplish hard tasks (like when carrying a large food item), but at other times act more individually\cite{feinerman2017individual}.
    This paper reviews the conditions for centralized or decentralized behavior within a colony, focusing on distributed behavior.
    Distributed behavior can be analyzed using alarm-response data in ants: A single alarmed ant will recruit other members of its colony, and the signal will propogate to some degree, eventually dying out.
    Depending on the strength of the signal, different networks of the colony are recruited.
    For example, centralized behavior is dependent on caste, and so, often the caretaker ants are harder to recruit during alarm response.
    Additionally, we are curious how quickly this information can propogate, as has been shown in bees \cite{gernat2018automated}. 
    Similarly, when retrieving food, cooperation is only necessary if the object is very large.
    It seems that the colony is distributed, but becomes more centralized when doing crucial tasks. 
    Degree of distributions is quantified by the clustering coefficient of the ant-network during alarm response.
    %% Ideally, the output of this project will be 
    This project constructs a mathematical model that predicts the clustering coefficient of an ant colony given information about alarm strength (Say, as a function of the number of alarmed scouts). 
    Possible further work will predict additional properties of the information-network.

    %% Another interesting idea is the notion of scale: The ants themselves are complex systems, and the colony as a whole is complex. 

%% For Nature, the abstract is really an introductory paragraph set
%% in bold type.  This paragraph must be ``fully referenced'' and
%% less than 180 words for Letters.  This is the thing that is
%% supposed to be aimed at people from other disciplines and is
%% arguably the most important part to getting your paper past the
%% editors.  End this paragraph with a sentence like ``Here we
%% show...'' or something similar.
\end{abstract}

\section{Introduction}

The universal is fundamentally distributed.
Each element, in essence, computes itself, without relying on other elements\cite{lloyd2006programming}.
However, elements in the universe can group (or "glom") together, forming local interacting clusters of independent elements.
In this view, local network-clusters of elements act as a coherent whole.
Also, even though the network is primarily distributed, certain networks display interesting highly-connected properties\cite{watts_strogatz_2011}.
Ant colonies display many interesting distributed and clustered behaviors.
First, individual ants are themselves chaotic. 
A single ant that has been disturbed will, when released, trace out an extremely "noisy" path, but one with abstract structure (it traces out its borders and secures its environment).
Now, ants themselves have brains -- their brains are also complex adaptive systems, and are not yet fully understood. 
%% With this view: One who says "The brain is like an ant colony" is saying "The brain is like a colony of smaller brains", "The brain is like a complex network of smaller brains"... what's a base brain, in this recursive scheme?
In a simplified view, we can view ants as uniform agents, perhaps agents without chaotic properties.
%% Interestingly, in Conway's Game of Life, there is no inherent randomness. Let's follow Von Neumann on this one: The only randomness should be in qubit simulations!.. or ...
Even if the ants themselves are not chaotic, the colony as a whole will act chaotic, because of the interaction between pieces.
However, the ants themselves are very much chaotic, complecting the analysis of this system further.
As a whole, the colony is capable of doing coordinated tasks -- there is, again, the emergence of a kind of Order from Chaos (in that the higher-level behavior is more easily described, but emerges from a complex ground-truth). 
%% Let us say that Order is easily described by a higher-level equation, where chaos is only described by the ground-truth quantum simulation of the system

%% Others have considered the distributed nature of the brain \cite{sporns_2004}

Misc citations: (Ant System) \cite{brabazon2017foraging}, \cite{brabazon2016distributed} \cite{1997} \cite{dorigo_maniezzo_colorni_1996}
Insects: \cite{heyman2017ants}, \cite{fonio2016locally}, \cite{gelblum2015ant}

\section{Problem(s)}

Given that all interesting natural computations are carried out in a network, we wish to know: 
"Are all computations carried out in the distributed network? Are some computations carried out in more restricted networks, while others recruit larger subsets?"
To what degree are particular activities centralized or decentralized?
What makes it necessary for an activity to be computed distributedly?
What makes it necessary for an activity to be computed centrally?

We might desire: "A computational study of the powers and limitations of natural distributed algorithms" Specifically, there is a need to "better understand the complexity and evolvability of distributed solutions to different qualities"
In other words: Why do natural systems (like ants) work this way?
How is it advantageous?
How is it disadvantageous?
Is it a necessary property of any intelligent system?

\subsection{Hypothesis:}
Distributed, adaptive nature IS a necessary property of any intelligent system.
In other words, it must be possible for the system to work in true parallel.
A truly capable system MUST be built as a complex network of complex parts, likewise composed, down to an arbitrary depth.
%% To test this hypothesis, we will create and compare existing cognitive systems, as well as ant colonies../??????????
[]

\section{Discussion Sections:}
\subsection{Architectural Discrepancy}
    Decentralized systems on a centralized Von-Neumann architecture? What about a network of Von-Neumann Architectures?
    [Cite Von Neumann, Seth Lloyd]
\subsection{Low level individual behavior}
    []
\subsection{High level emergence}
    Ants are capable of building complex structures with no blueprint, communicating with only local information.
    []
\subsection{A Complex Adaptive System built from Complex Adaptive Systems}
    []
\subsection{Blending of individual and collective cognition}
    []
\subsection{Review of Implementations in Algorithmic Biomimicry}
    \subsubsection{Ant Systems}
        "Ant systems" have been deployed to solve several interesting problems. 
        However, these software implementations are highly idealized models, especially when considering the chaotic behavior present even in individual ants. 
        There is a large amount of behavior unaccounted for, and it is highly likely that accounting for this behavior will increase the effectiveness of biomimetic algorithms.
        [Cite Ant Systems Papers]

    \subsubsection{Abstract Agent Based Systems}
        Certain cognitive architectures are also inspired by the analogy of an ant colony. 
        Similarly, these architectures can miss important features of real-world complex adaptive systems.
        It is an open question: How distributed is one of these architectures? When does it act collectively? Individually? How does this mimic the behavior of an ant colony?
        I will perform perturbations to the cognitive architecture "copycat," to attempt to answer this.
        [Cite Alexandre Linhares, Self, Copycat, Hofstadter]

    \subsubsection{Avoiding Combinatorial Explosion}
        Through the use of distributed agents, an algorithm the mimics distributed social insects is able to avoid combinatorial explosion.
        Combinatorial explosion is a common phenomena which limits the capabilities of many artificial intelligence systems.
        For instance, the original super-human chess algorithm used "Alpha-Beta" search with a few fine-tuned tricks, and simply exploited large computational resources to calculate several moves in advance.
        However, it is extremely unlikely that human chess players play this way, and even the worlds most powerful computers are fundamentally limited in the depth that they can consider (For, when considering all possible games, the number of possibilities is larger than, say, the number of atoms in the universe).
        Human-like algorithms have been proposed, which attempt to avoid enumerating potential game states. 
        By avoiding explicit enumeration, they no longer run in exponential time.
        [Cite Alexandre Linhares, Patrick Henry Winston]
\subsection{}
    []

Brains:
\cite{koch_1999}
Insects:
\cite{fewell_2003}

\bibliography{sources/insects,sources/brains,sources/ant_system,sources/networks,sources/misc}

\end{document}
