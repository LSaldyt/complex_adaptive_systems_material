%% Template for a preprint Letter or Article for submission
%% to the journal Nature.
%% Written by Peter Czoschke, 26 February 2004
%% Edited by Lucas Saldyt, September 11th, 2018

\documentclass{nature}

%% make sure you have the nature.cls and naturemag.bst files where
%% LaTeX can find them

\bibliographystyle{naturemag}

\title{Biomimicry of Distributed Cognition in Social Insects}

\author{Lucas Saldyt}

\begin{document}

\maketitle

\begin{affiliations}
 \item Arizona State University
\end{affiliations}

\begin{abstract}
    This is my abstract
%% For Nature, the abstract is really an introductory paragraph set
%% in bold type.  This paragraph must be ``fully referenced'' and
%% less than 180 words for Letters.  This is the thing that is
%% supposed to be aimed at people from other disciplines and is
%% arguably the most important part to getting your paper past the
%% editors.  End this paragraph with a sentence like ``Here we
%% show...'' or something similar.
\end{abstract}

\section{Introduction}

The universal is fundamentally distributed.
Each element, in essence, computes itself, without relying on other elements [Seth Lloyd].
However, elements in the universe can group (or "glom") together, forming local interacting clusters of independent elements.
In this view, local network-clusters of elements act as a coherent whole.
For concreteness, we will consider the cluster that is an ant colony. 
Ant colonies display many interesting distributed and clustered behaviors.
First, individual ants are themselves chaotic. 
For instance, a single ant that has been disturbed will, when released, trace out an extremely "noisy" path, but one with abstract structure (it traces out its borders and secures its environment).
Now, ants themselves have brains -- their brains are also complex adaptive systems, and are not yet fully understood. 
%% With this view: One who says "The brain is like an ant colony" is saying "The brain is like a colony of smaller brains", "The brain is like a complex network of smaller brains"... what's a base brain, in this recursive scheme?
In a simplified view, we can view ants as uniform agents, perhaps agents without chaotic properties.
%% Interestingly, in Conway's Game of Life, there is no inherent randomness. Let's follow Von Neumann on this one: The only randomness should be in qubit simulations!.. or ...
Even if the ants themselves are not chaotic, the colony as a whole will act chaotic, because of the interaction between pieces.
However, the ants themselves are very much chaotic, complecting the analysis of this system further.
As a whole, the colony is capable of doing coordinated tasks -- there is, again, the emergence of a kind of Order from Chaos (in that the higher-level behavior is more easily described, but emerges from a complex ground-truth). 
%% Let us say that Order is easily described by a higher-level equation, where chaos is only described by the ground-truth quantum simulation of the system

\section{Problem(s)}

Given that all interesting natural computations are carried out in a network, we wish to know: 
"Are all computations carried out in the distributed network? Are some computations carried out in more restricted networks, while others recruit larger subsets?"
To what degree are particular activities centralized or decentralized?
What makes it necessary for an activity to be computed distributedly?
What makes it necessary for an activity to be computed centrally?

We might desire: "A computational study of the powers and limitations of natural distributed algorithms" Specifically, there is a need to "better understand the complexity and evolvability of distributed solutions to different qualities"
In other words: Why do natural systems (like ants) work this way?
How is it advantageous?
How is it disadvantageous?
Is it a necessary property of any intelligent system?

\subsection{Hypothesis:}
Distributed, adaptive nature IS a necessary property of any intelligent system.
In other words, it must be possible for the system to work in true parallel.
A truly capable system MUST be built as a complex network of complex parts, likewise composed, down to an arbitrary depth.
%% To test this hypothesis, we will create and compare existing cognitive systems, as well as ant colonies../??????????
[]

\section{Discussion Sections:}
\subsection{Architectural Discrepancy}
    Decentralized systems on a centralized Von-Neumann architecture? What about a network of Von-Neumann Architectures?
    [Cite Von Neumann, Seth Lloyd]
\subsection{Low level individual behavior}
    []
\subsection{High level emergence}
    Ants are capable of building complex structures with no blueprint, communicating with only local information.
    []
\subsection{A Complex Adaptive System built from Complex Adaptive Systems}
    []
\subsection{Blending of individual and collective cognition}
    []
\subsection{Review of Implementations in Algorithmic Biomimicry}
    \subsubsection{Ant Systems}
        "Ant systems" have been deployed to solve several interesting problems. 
        However, these software implementations are highly idealized models, especially when considering the chaotic behavior present even in individual ants. 
        There is a large amount of behavior unaccounted for, and it is highly likely that accounting for this behavior will increase the effectiveness of biomimetic algorithms.
        [Cite Ant Systems Papers]

    \subsubsection{Abstract Agent Based Systems}
        Certain cognitive architectures are also inspired by the analogy of an ant colony. 
        Similarly, these architectures can miss important features of real-world complex adaptive systems.
        It is an open question: How distributed is one of these architectures? When does it act collectively? Individually? How does this mimic the behavior of an ant colony?
        I will perform perturbations to the cognitive architecture "copycat," to attempt to answer this.
        [Cite Alexandre Linhares, Self, Copycat, Hofstadter]

    \subsubsection{Avoiding Combinatorial Explosion}
        Through the use of distributed agents, an algorithm the mimics distributed social insects is able to avoid combinatorial explosion.
        Combinatorial explosion is a common phenomena which limits the capabilities of many artificial intelligence systems.
        For instance, the original super-human chess algorithm used "Alpha-Beta" search with a few fine-tuned tricks, and simply exploited large computational resources to calculate several moves in advance.
        However, it is extremely unlikely that human chess players play this way, and even the worlds most powerful computers are fundamentally limited in the depth that they can consider (For, when considering all possible games, the number of possibilities is larger than, say, the number of atoms in the universe).
        Human-like algorithms have been proposed, which attempt to avoid enumerating potential game states. 
        By avoiding explicit enumeration, they no longer run in exponential time.
        [Cite Alexandre Linhares, Patrick Henry Winston]
\subsection{}
    []

%% Then the body of the main text appears after the intro paragraph.
%% Figure environments can be left in place in the document.
%% \verb|\includegraphics| commands are ignored since Nature wants
%% the figures sent as separate files and the captions are
%% automatically moved to the end of the document (they are printed
%% out with the \verb|\end{document}| command. However, tables must
%% be manually moved to the end of the document, after the addendum.
%% 
%% Citation of Einstein's paper \cite{Einstein}.
%% 
%% \begin{figure}
%% %%%\includegraphics{something} % this command will be ignored
%% \caption{Each figure legend should begin with a brief title for
%% the whole figure and continue with a short description of each
%% panel and the symbols used. For contributions with methods
%% sections, legends should not contain any details of methods, or
%% exceed 100 words (fewer than 500 words in total for the whole
%% paper). In contributions without methods sections, legends should
%% be fewer than 300 words (800 words or fewer in total for the whole
%% paper).}
%% \end{figure}
%% 
%% \section*{Another Section}
%% 
%% Sections can only be used in Articles.  Contributions should be
%% organized in the sequence: title, text, methods, references,
%% Supplementary Information line (if any), acknowledgements,
%% interest declaration, corresponding author line, tables, figure
%% legends.
%% 
%% Spelling must be British English (Oxford English Dictionary)
%% 
%% In addition, a cover letter needs to be written with the
%% following:
%% \begin{enumerate}
%%  \item A 100 word or less summary indicating on scientific grounds
%% why the paper should be considered for a wide-ranging journal like
%% \textsl{Nature} instead of a more narrowly focussed journal.
%%  \item A 100 word or less summary aimed at a non-scientific audience,
%% written at the level of a national newspaper.  It may be used for
%% \textsl{Nature}'s press release or other general publicity.
%%  \item The cover letter should state clearly what is included as the
%% submission, including number of figures, supporting manuscripts
%% and any Supplementary Information (specifying number of items and
%% format).
%%  \item The cover letter should also state the number of
%% words of text in the paper; the number of figures and parts of
%% figures (for example, 4 figures, comprising 16 separate panels in
%% total); a rough estimate of the desired final size of figures in
%% terms of number of pages; and a full current postal address,
%% telephone and fax numbers, and current e-mail address.
%% \end{enumerate}
%% 
%% See \textsl{Nature}'s website
%% (\texttt{http://www.nature.com/nature/submit/gta/index.html}) for
%% complete submission guidelines.
%% 
%% \begin{methods}
%% Put methods in here.  If you are going to subsection it, use
%% \verb|\subsection| commands.  Methods section should be less than
%% 800 words and if it is less than 200 words, it can be incorporated
%% into the main text.
%% 
%% \subsection{Method subsection.}
%% 
%% Here is a description of a specific method used.  Note that the
%% subsection heading ends with a full stop (period) and that the
%% command is \verb|\subsection{}| not \verb|\subsection*{}|.
%% 
%% \end{methods}
%% 
%% %% Put the bibliography here, most people will use BiBTeX in
%% %% which case the environment below should be replaced with
%% %% the \bibliography{} command.
%% 
%% % \begin{thebibliography}{1}
%% % \bibitem{dummy} Articles are restricted to 50 references, Letters
%% % to 30.
%% % \bibitem{dummyb} No compound references -- only one source per
%% % reference.
%% % \end{thebibliography}
%% 
%% \bibliographystyle{naturemag}
%% \bibliography{sample}
%% 
%% 
%% %% Here is the endmatter stuff: Supplementary Info, etc.
%% %% Use \item's to separate, default label is "Acknowledgements"
%% 
%% \begin{addendum}
%%  \item Put acknowledgements here.
%%  \item[Competing Interests] The authors declare that they have no
%% competing financial interests.
%%  \item[Correspondence] Correspondence and requests for materials
%% should be addressed to A.B.C.~(email: myaddress@nowhere.edu).
%% \end{addendum}
%% 
%% %%
%% %% TABLES
%% %%
%% %% If there are any tables, put them here.
%% %%
%% 
%% \begin{table}
%% \centering
%% \caption{This is a table with scientific results.}
%% \medskip
%% \begin{tabular}{ccccc}
%% \hline
%% 1 & 2 & 3 & 4 & 5\\
%% \hline
%% aaa & bbb & ccc & ddd & eee\\
%% aaaa & bbbb & cccc & dddd & eeee\\
%% aaaaa & bbbbb & ccccc & ddddd & eeeee\\
%% aaaaaa & bbbbbb & cccccc & dddddd & eeeeee\\
%% 1.000 & 2.000 & 3.000 & 4.000 & 5.000\\
%% \hline
%% \end{tabular}
%% \end{table}

\end{document}
